\chapter{Preemption and Priorities}

In Chapter~\ref{chap:contexts} we built a cooperative scheduler: processes
yielded voluntarily. This worked, but a process that forgot to call
\texttt{sched\_yield()} could monopolize the CPU forever. In this chapter,
EmbryOS gains a hardware timer and turns cooperative multitasking into
\emph{preemptive} scheduling. We also introduce priorities so that background
tasks run only when nothing more important is ready.

\section{Trap Entry}

When any interrupt or exception occurs, the CPU jumps to the address stored in
the \texttt{mtvec} register. EmbryOS installs \texttt{\_trap\_handler}, shown in
Listing~\ref{lst:trap_s}, as a single entry point. It saves all general registers
and calls \texttt{software\_trap\_handler()} in \texttt{clint.c}. The handler
decides what to do—timer interrupt, system call, or fault—then returns to the
same program counter via \texttt{mret}.

\begin{figure}[H]
\centering
\lstinputlisting[
  style=oscode,
  language={[x86masm]Assembler},
  caption={Trap entry and exit, \texttt{trap.s}},
  label={lst:trap_s}
]{../code/chapter4/trap_short.s}
\end{figure}

\section{Core Local Interruptor}

The \emph{CLINT} (Core Local Interruptor) connects each CPU core to its local
timer. Its driver in Listings~\ref{lst:clint_c}–\ref{lst:clint_h} installs the
trap handler, registers a C handler for each interrupt type, and provides simple
functions to enable and disable interrupts globally.

\begin{figure}[H]
\centering
\lstinputlisting[
  style=oscode,
  language=C,
  caption={Interrupt controller, \texttt{clint.c}},
  label={lst:clint_c}
]{../code/chapter4/clint.c}
\end{figure}

\begin{figure}[H]
\centering
\lstinputlisting[
  style=oscode,
  language=C,
  caption={Interrupt controller interface, \texttt{clint.h}},
  label={lst:clint_h}
]{../code/chapter4/clint.h}
\end{figure}

\section{Machine Timer}

Every RISC-V core includes a 64-bit counter (\texttt{mtime}) and a compare
register (\texttt{mtimecmp}). When the counter reaches the compare value, a
machine-timer interrupt is raised. EmbryOS programs this timer in
\texttt{mtime.c}. Each interrupt calls the scheduler and sets the next alarm,
creating periodic preemption.

\begin{figure}[H]
\centering
\lstinputlisting[
  style=oscode,
  language=C,
  caption={Machine timer, \texttt{mtime.c}},
  label={lst:mtime_c}
]{../code/chapter4/mtime.c}
\end{figure}

\begin{figure}[H]
\centering
\lstinputlisting[
  style=oscode,
  language=C,
  caption={Timer interface, \texttt{mtime.h}},
  label={lst:mtime_h}
]{../code/chapter4/mtime.h}
\end{figure}

\section{Prioritized Scheduler}

The scheduler now maintains three run queues: high, medium, and low. New
processes enter at the highest level. Each time a process exhausts its quantum,
it is demoted to the next queue. This simple aging policy ensures that every
process runs, but interactive ones get more CPU time.

\begin{figure}[H]
\centering
\lstinputlisting[
  style=oscode,
  language=C,
  caption={Priority scheduler, \texttt{sched.c}},
  label={lst:sched_c}
]{../code/chapter4/sched.c}
\end{figure}

\begin{figure}[H]
\centering
\lstinputlisting[
  style=oscode,
  language=C,
  caption={Scheduler interface, \texttt{sched.h}},
  label={lst:sched_h}
]{../code/chapter4/sched.h}
\end{figure}

\section{Demonstration: Preemption in Action}

The demonstration program (Listing~\ref{lst:hello_c}) installs a timer handler,
creates four processes, and enables interrupts. Unlike Chapter~\ref{chap:contexts},
no process ever calls \texttt{sched\_yield()}. The timer interrupt now drives
context switches automatically, proving that preemption works.

\begin{figure}[H]
\centering
\lstinputlisting[
  style=oscode,
  language=C,
  caption={Preemptive demo, \texttt{hello.c}},
  label={lst:hello_c}
]{../code/chapter4/hello.c}
\end{figure}

\section*{Projects}

\begin{enumerate}
  \item \textbf{Different quanta.} Change \texttt{QUANTUM} and observe the
        smoothness of animation.
  \item \textbf{Process aging.} Add a rule that moves idle processes back to a
        higher priority.
  \item \textbf{Other interrupts.} Extend \texttt{clint.c} to handle software
        and external interrupts.
\end{enumerate}
