\chapter{Seeing Is Believing}

In Chapter~1, we printed ``Hello~World'' by writing to a UART register.  In this
chapter we will take the next step and \emph{see} something appear on a screen.
To do this, we will ask QEMU to emulate a simple framebuffer device, so that
our kernel can draw pixels directly into memory.

\section{From UART to Framebuffer}

A \textbf{framebuffer} is simply a block of memory where each 32-bit word
represents one pixel on the screen.  When your program writes color values into
this memory, the display hardware (or QEMU) updates the corresponding pixels on
the screen.

QEMU can emulate such a framebuffer with the \texttt{ramfb} device.  This device
is extremely simple: it just exposes a memory region, and whatever bytes you
write there become visible in a window on your host system.

\section{Running QEMU with a Screen}

We can reuse the same build environment from Chapter~1, but we will run QEMU
with an additional device and a graphical display:

\begin{lstlisting}[style=oscode,language=bash,caption={Running QEMU with a RAM framebuffer}]
$ qemu-system-riscv32 -machine virt -device ramfb -display sdl \
    -bios hello.bin
\end{lstlisting}

The \texttt{-device ramfb} option adds a framebuffer device, and
\texttt{-display sdl} tells QEMU to open a window to show it.  You will still
see UART output in your terminal if you add \texttt{-serial mon:stdio}.

\section{Drawing to the Framebuffer}

For simplicity, we start by assuming that QEMU places the framebuffer at address
\texttt{0x40000000}, with a resolution of 640\,$\times$\,480 pixels, and that
each pixel is a 32-bit ARGB value (8 bits per color channel).  These defaults
work well for the \texttt{ramfb} device.

The following C code paints the entire screen green:

\begin{figure}[H]
\centering
\begin{minipage}{0.9\textwidth}
\begin{lstlisting}[style=oscode,language=C,caption={Drawing a solid color on the framebuffer},label={lst:framebuffer}]
#define FRAMEBUFFER ((volatile unsigned int *) 0x40000000)
#define WIDTH   640
#define HEIGHT  480

void draw_screen(unsigned int color) {
    for (int y = 0; y < HEIGHT; y++) {
        for (int x = 0; x < WIDTH; x++) {
            FRAMEBUFFER[y * WIDTH + x] = color;
        }
    }
}

int main() {
    draw_screen(0x0000FF00);   // ARGB: 00RRGGBB --> bright green
    while (1) ;
}
\end{lstlisting}
\end{minipage}
\end{figure}

When you build and run this version of EmbryOS, QEMU opens a window filled with
green.  Congratulations—your operating system just drew its first pixels!

\section{Understanding the Address}

In a real system, the framebuffer might be located elsewhere in physical memory
and described in the \textbf{device tree}.  QEMU also passes such a device tree
to the guest, containing an entry like:

\begin{lstlisting}[style=oscode,language={}]
framebuffer@40000000 {
    compatible = "ramfb";
    reg = <0x0 0x40000000 0x0 0x96000>;
};
\end{lstlisting}

Later, we will learn how to parse this tree so that the kernel can discover the
framebuffer address automatically.  For now, we simply hardcode it.

\section{Projects}

\begin{enumerate}
  \item \textbf{Change the color.}  
        Modify \texttt{draw\_screen()} to fill the screen with red, blue, or a
        gradient of your own design.
  \item \textbf{Draw a pattern.}  
        Replace the solid fill with horizontal or vertical stripes by varying
        the color every few rows or columns.
  \item \textbf{Simple animation.}  
        Alternate colors in a loop, adding a short delay between frames.  You
        will see the screen flicker as EmbryOS repeatedly redraws it.
  \item \textbf{Optional challenge: parse the device tree.}  
        Look at the \texttt{ramfb} node in the device tree and try to extract
        the framebuffer address and size instead of hardcoding them.
\end{enumerate}
