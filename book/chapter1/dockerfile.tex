\section{The Dockerfile}

Before we write any kernel code, we need a consistent build and run environment.
To avoid platform differences (macOS vs. Linux, Intel vs. ARM), we will use
\texttt{Docker}.  Docker lets us define an entire Linux environment in a single
file—called a \texttt{Dockerfile}—that anyone can build and run.

Listing~\ref{lst:dockerfile} shows the complete Dockerfile for EmbryOS.  It
creates a container that installs all necessary tools: a RISC\textendash V
cross-compiler, the QEMU emulator, and a simple text editor (\texttt{nano}).
Once built, this container lets you compile and run EmbryOS on any machine.

% --- Dockerfile listing float ---
\begin{figure}[H]
\centering
\begin{minipage}{0.95\textwidth}
\begin{lstlisting}[style=oscode,language={},caption={Dockerfile for building and running EmbryOS on macOS/arm64},label={lst:dockerfile}]
# ------------------------------------------------------------
# Dockerfile for building and running EmbryOS on macOS/arm64
# ------------------------------------------------------------
FROM ubuntu:22.04

ENV DEBIAN_FRONTEND=noninteractive
ENV EMBRYOS=/workspace
WORKDIR $EMBRYOS

# Install build tools, RISC-V toolchain, QEMU, ...
RUN apt-get update && apt-get install -y --no-install-recommends \
    build-essential \
    gcc \
    make \
    git \
    wget \
    zip \
    unzip \
    gdb-multiarch \
    ca-certificates \
    nano \
    && apt-get clean && rm -rf /var/lib/apt/lists/*

# Clone EmbryOS
RUN git clone https://github.com/rvanren/EmbryOS.git

# Setup the RISC-V compiler
RUN wget https://github.com/xpack-dev-tools/riscv-none-elf-gcc-xpack/releases/download/v14.2.0-3/xpack-riscv-none-elf-gcc-14.2.0-3-linux-arm64.tar.gz && \
    tar -xzf xpack-riscv-none-elf-gcc-14.2.0-3-linux-arm64.tar.gz
ENV PATH="${PATH}:${EMBRYOS}/xpack-riscv-none-elf-gcc-14.2.0-3/bin"

# Setup QEMU
RUN wget https://github.com/xpack-dev-tools/qemu-riscv-xpack/releases/download/v8.2.2-1/xpack-qemu-riscv-8.2.2-1-linux-arm64.tar.gz && \
    tar -xzf xpack-qemu-riscv-8.2.2-1-linux-arm64.tar.gz
ENV PATH="${PATH}:${EMBRYOS}/xpack-qemu-riscv-8.2.2-1/bin"

CMD ["/bin/bash"]
\end{lstlisting}
\end{minipage}
\end{figure}

To build the container, run:

\begin{lstlisting}[style=oscode,language=bash]
$ docker build -t embryos-riscv .
\end{lstlisting}

The \texttt{-t} flag names the image \texttt{embryos-riscv}.  
Once the build finishes, you can start a container using:

\begin{lstlisting}[style=oscode,language=bash]
$ docker run -it --rm embryos-riscv
\end{lstlisting}

This command:
\begin{itemize}
  \item \texttt{-it} starts an interactive shell,
  \item \texttt{--rm} removes the container when you exit,
  \item \texttt{-v \$(pwd):/workspace} mounts your current directory inside the container.
\end{itemize}

When the shell prompt appears, you are inside a clean Ubuntu system with all the
tools installed.  You can edit files using \texttt{nano}, compile with
\texttt{make}, and run QEMU directly.

\bigskip
\noindent
This environment will be reused throughout the book; once you’ve built this
Docker image, you’re ready for every chapter that follows.
