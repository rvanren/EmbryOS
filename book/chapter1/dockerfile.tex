\section{The Dockerfile}

Before we write any kernel code, we need a consistent build and run environment.
To avoid platform differences (macOS vs. Linux, Intel vs. ARM), we will use
\texttt{Docker}.  Docker lets us define an entire Linux environment in a single
file—called a \texttt{Dockerfile}—that anyone can build and run.

Listing~\ref{lst:dockerfile} shows the complete Dockerfile for EmbryOS.  It
creates a container that installs all necessary tools: a RISC\textendash V
cross-compiler, the QEMU emulator, and a simple text editor (\texttt{nano}).
Once built, this container lets you compile and run EmbryOS on any machine.

% --- Dockerfile listing float ---
\begin{figure}[H]
\centering
\begin{minipage}{0.95\textwidth}
\lstinputlisting[style=oscode,language={},caption={Dockerfile for building and running EmbryOS on macOS/arm64},label={lst:dockerfile}]{../code/Dockerfile}
\end{minipage}
\end{figure}

To build the container, run:

\begin{lstlisting}[style=oscode,language=bash]
$ docker build --no-cache -t embryos-riscv .
\end{lstlisting}

The \texttt{-t} flag names the image \texttt{embryos-riscv}.  
Once the build finishes, you can start a container using:

\begin{lstlisting}[style=oscode,language=bash]
$ docker run -it --rm embryos-riscv
\end{lstlisting}

This command:
\begin{itemize}
  \item \texttt{-it} starts an interactive shell,
  \item \texttt{--rm} removes the container when you exit,
  \item \texttt{-v \$(pwd):/workspace} mounts your current directory inside the container.
\end{itemize}

When the shell prompt appears, you are inside a clean Ubuntu system with all the
tools installed.  You can edit files using \texttt{nano}, compile with
\texttt{make}, and run QEMU directly.

\bigskip
\noindent
This environment will be reused throughout the book; once you’ve built this
Docker image, you’re ready for every chapter that follows.
