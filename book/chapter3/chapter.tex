\chapter{Processes}
\label{chap:contexts}

In Chapter~\ref{chap:textscreen} we gave EmbryOS a logical screen and learned
to animate text.  But a single thread of control limits what the system can do
at once.  This chapter introduces \emph{contexts}: independent stacks and
register sets that can share the CPU cooperatively.  Each context represents a
separate \emph{process}.  The kernel saves one process's state and restores
another's, giving the illusion of concurrent activity.

\section{What Is a Context?}

A context is the minimal information needed to resume a computation later:
registers and the stack pointer.  On RISC-V, this includes the program counter
(\texttt{ra}) and all callee-saved registers.  EmbryOS performs a context
switch entirely in assembly to ensure nothing is lost.

\begin{figure}[H]
\centering
\begin{minipage}{0.75\textwidth}
\lstinputlisting[
  style=oscode,
  language={[x86masm]Assembler},
  caption={Context switch code in \texttt{ctx.s}},
  label={lst:ctx_s}
]{../code/chapter3/ctx_short.s}
\end{minipage}
\end{figure}

The functions exported by this module, declared in
Listing~\ref{lst:ctx_h}, are:

\begin{itemize}
  \item \texttt{ctx\_switch(\&old\_sp, new\_sp)} ---
        save the current context (its callee-saved registers and stack
        pointer) and restore another.
  \item \texttt{ctx\_start(\&save\_sp, new\_sp, entry)} ---
        create a brand-new context whose first instruction will call
        \texttt{entry()}.
\end{itemize}

\begin{figure}[H]
\centering
\begin{minipage}{0.6\textwidth}
\lstinputlisting[
  style=oscode,
  language=C,
  caption={Context-switch interface in \texttt{ctx.h}},
  label={lst:ctx_h}
]{../code/chapter3/ctx.h}
\end{minipage}
\end{figure}

\section{Processes and the Process Table}

EmbryOS keeps track of active processes in a small table.  Each entry combines
a \emph{process control block} (PCB) and its stack in a single 4~KiB page, as
shown in Figure~\ref{fig:pcb_layout}.  The top of the page serves as the stack
and the base holds metadata.

\begin{figure}[H]
\centering
\begin{verbatim}
+---------------------------+  <--- high memory
|        process stack       |
|             ...             |
|-----------------------------|
| struct pcb {                |
|     void *sp;               |
|     int pid, state;         |
|     struct rect area;       |
| };                          |
+-----------------------------+  <--- low memory
\end{verbatim}
\caption{Layout of one process page.}
\label{fig:pcb_layout}
\end{figure}

\noindent
The PCB records not only scheduling state but also a \texttt{rect}
describing the region of the screen that the process ``owns.''

\begin{figure}[H]
\centering
\begin{minipage}{0.9\textwidth}
\lstinputlisting[
  style=oscode,
  language=C,
  caption={Process management in \texttt{process.c}},
  label={lst:process_c}
]{../code/chapter3/process.c}
\end{minipage}
\end{figure}

Each process has local coordinates within its rectangle.
The function \texttt{process\_put()} translates these local coordinates into
global screen positions before calling \texttt{screen\_move()} and
\texttt{screen\_put()}.  This ensures that processes remain confined to their
own regions, much as modern operating systems isolate processes in their own
address spaces.

\section{Cooperative Scheduling}

All processes share a single CPU.  They take turns voluntarily by calling
\texttt{proc\_yield()}, which performs a \texttt{ctx\_switch()} from the
current stack to the next process's stack.  The scheduler follows a simple
round-robin policy.

Because there are no interrupts yet, EmbryOS implements
\emph{cooperative multitasking}: a process keeps the CPU until it explicitly
yields.  The next chapter will introduce timer interrupts to make this
preemptive.

\section{Demonstration: Four Windows}

The demonstration program in Listing~\ref{lst:hello_c} creates four processes,
each with its own window on the 80$\times$24 screen.  Each process runs a small
animation that prints a moving letter in its region.

\begin{figure}[H]
\centering
\begin{minipage}{0.9\textwidth}
\lstinputlisting[
  style=oscode,
  language=C,
  caption={Multitasking demo in \texttt{hello.c}},
  label={lst:hello_c}
]{../code/chapter3/hello.c}
\end{minipage}
\end{figure}

When run under QEMU with \texttt{-nographic}, the display looks something like:

\begin{verbatim}
AAAAA                      BBBBB
AAAAA                      BBBBB
AAAAA                      BBBBB
                 CCCCC
                 CCCCC
                 CCCCC
DDDDD
DDDDD
DDDDD
\end{verbatim}

Each process draws independently, yet the system cycles between them,
preserving their stacks and registers between switches.

\section*{Projects}

\begin{enumerate}
  \item \textbf{More processes.}
        Increase \texttt{MAX\_PROCESSES} and experiment with six or eight
        windows.  How does the system behave?
  \item \textbf{Process termination.}
        Add a new state to the PCB for ``terminated'' and modify the scheduler
        to skip finished processes.
  \item \textbf{Dynamic regions.}
        Let each process move its rectangle periodically to simulate window
        management.
\end{enumerate}
