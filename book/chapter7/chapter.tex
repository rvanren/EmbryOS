\chapter{User Mode and System Calls}
\label{chap:user-mode}

In Chapter~\ref{chap:termination}, processes could already create new processes
and terminate using the \texttt{!} and \texttt{.} keys, but all of that still
happened entirely in \emph{kernel mode}.  In this chapter, we introduce
\emph{user mode} and move process management behind a \emph{system-call}
interface.  The same commands now work by trapping into the kernel through
\texttt{ecall}, marking the transition from a monolithic kernel to one that
supports protected user processes.

\section{Learning Goals}

\begin{itemize}
  \item Understand RISC-V privilege levels and the trap mechanism between
        user and kernel mode.
  \item Implement system calls for process creation, termination, and I/O.
  \item Observe how user code invokes kernel services using the \texttt{ecall}
        instruction.
\end{itemize}

\section{Overview}

User code now runs in a lower privilege level.  When it needs to interact with
the kernel—printing to the screen, reading a key, spawning a new process, or
exiting—it executes an \texttt{ecall} instruction.  
The hardware traps into the kernel, which inspects register~\texttt{a7} to
identify the requested service and executes it safely on behalf of the user.

From the user's perspective, nothing changes: typing \texttt{!} still spawns a
new process and typing \texttt{.} still exits.  What has changed is that these
actions now go through a carefully controlled user–kernel boundary.

\section{Context Setup for User Mode}

The assembly file \texttt{ctx.s} defines context switching and user-mode entry.
The function \texttt{ctx\_user\_setup()} configures Physical Memory Protection
(PMP) to allow user code to run within a limited memory region.  The
\texttt{mret} instruction then switches to user mode.

\begin{figure}[H]
\centering
\lstinputlisting[
  style=oscode,
  language={[x86masm]Assembler},
  caption={Context management and user-mode entry, \texttt{ctx.s}},
  label={lst:ctx_s}
]{../code/chapter7/ctx_short.s}
\end{figure}

\section{System Calls}

The system-call handler dispatches requests based on the call number in
register~\texttt{a7}.  Two new calls, \texttt{SYS\_EXIT} and
\texttt{SYS\_SPAWN}, provide process control, while \texttt{SYS\_GET} and
\texttt{SYS\_PUT} handle I/O.

\begin{figure}[H]
\centering
\lstinputlisting[
  style=oscode,
  language=C,
  caption={System-call handler with user-mode process control, \texttt{syscall.c}},
  label={lst:syscall_c}
]{../code/chapter7/syscall.c}
\end{figure}

Their numeric assignments appear in \texttt{syscall.h}.

\begin{figure}[H]
\centering
\lstinputlisting[
  style=oscode,
  language=C,
  caption={System-call numbers and prototypes, \texttt{syscall.h}},
  label={lst:syscall_h}
]{../code/chapter7/syscall.h}
\end{figure}

\section{User Library}

The user library \texttt{syslib.h} provides inline wrappers that hide the raw
\texttt{ecall} instructions.  Applications can now call functions such as
\texttt{user\_get()}, \texttt{user\_put()}, \texttt{user\_spawn()}, and
\texttt{user\_exit()} as ordinary C functions.  Each wrapper places arguments in
the appropriate registers and executes \texttt{ecall}.

\begin{figure}[H]
\centering
\lstinputlisting[
  style=oscode,
  language=C,
  caption={User-level system-call stubs, \texttt{syslib.h}},
  label={lst:syslib_h}
]{../code/chapter7/syslib.h}
\end{figure}

\section{Demonstration Program}

The demonstration (\texttt{hello.c}) is structurally similar to the previous
chapter but now runs its tasks entirely in user mode.  Both \texttt{taskA()} and
\texttt{taskB()} use user-level system-call wrappers to perform I/O and manage
processes.

When running, typing an exclamation mark~(\texttt{!}) spawns a new
\texttt{taskA()} process by invoking \texttt{user\_spawn()}, while typing a
period~(\texttt{.}) calls \texttt{user\_exit()} to terminate the current task.
The kernel mediates all of these requests through the system-call interface.

\begin{figure}[H]
\centering
\lstinputlisting[
  style=oscode,
  language=C,
  caption={Interactive demo in user mode using system calls, \texttt{hello.c}},
  label={lst:hello_c}
]{../code/chapter7/hello.c}
\end{figure}

\section{Discussion}

\paragraph{Continuity.}
The system behaves the same as in Chapter~\ref{chap:termination}, but the
architecture has changed fundamentally: user tasks now operate at a lower
privilege level, and all I/O or process management crosses into the kernel via
system calls.

\paragraph{Protection.}
While memory is still shared, privilege separation ensures that user code cannot
execute kernel instructions directly.  This is the first step toward true
isolation.

\paragraph{Toward per-process user stacks.}
Currently, user programs execute on kernel stacks for simplicity.  In the next
chapter, we will allocate separate user stacks to give each process its own
complete user context.

\section*{Projects}

\begin{enumerate}
  \item Add a new system call \texttt{SYS\_TIME} that returns the current timer
        value from user mode.
  \item Modify \texttt{user\_spawn()} to accept parameters such as color or
        position.
  \item Extend the system-call interface with an explicit \texttt{yield()}
        call for cooperative multitasking.
\end{enumerate}
