\chapter{User Mode and System Calls}
\label{chap:syscalls}

In Chapter~\ref{chap:blocking-io} we built an interrupt-driven kernel that supported
preemption and blocking I/O entirely within kernel mode.  
In this chapter we extend the kernel with the ability to execute application
code in \emph{user mode}, and to transfer control to the kernel through
\emph{system calls}.  
This introduces privilege separation and the first simple user–kernel interface.

\section{Learning Goals}

\begin{itemize}
  \item Understand RISC-V privilege levels and the role of the \texttt{mstatus} and \texttt{mepc} registers.
  \item Enter user mode safely using the \texttt{mret} instruction.
  \item Implement the \texttt{ecall} instruction to trap into the kernel.
  \item Provide a small system-call interface for I/O operations.
\end{itemize}

\section{Overview}

Up to this point, all code has executed in machine mode.  
We now introduce user mode so that application code executes with reduced privileges.  
When user code needs a service (for example, to print or read a character), it executes the
\texttt{ecall} instruction.  The hardware then traps into the kernel, which
examines the trap frame and dispatches to the appropriate service routine.

At this stage, user programs still share the same address space and kernel stack as the kernel.
Separate user stacks and address spaces will be introduced in Chapter~\ref{chap:picload}.

\section{Context Switching and User Entry}
\label{sec:ctx}

The file \texttt{ctx.s} contains two assembly routines:
\texttt{ctx\_switch}, which saves and restores callee-saved registers, and
\texttt{ctx\_start}, which switches to a new stack and begins execution of a
function in user mode.

\begin{figure}[H]
\centering
\lstinputlisting[
  style=oscode,
  language={[x86masm]Assembler},
  caption={Context management and entry into user mode, \texttt{ctx.s}},
  label={lst:ctx_s}
]{../code/chapter6/ctx_short.s}
\end{figure}

The helper \texttt{ctx\_user\_setup()} programs the Physical Memory Protection (PMP)
registers so that user code can safely access kernel memory for now.
It configures PMP~0 to cover 4\,MiB around \texttt{0x80000000}, enabling read, write,
and execute permissions.

\paragraph{Privilege control.}
To enter user mode, we clear the MPP (machine previous privilege) bits in
\texttt{mstatus} and set \texttt{mepc} to the user entry point.
The \texttt{mret} instruction restores privilege and begins user execution.

\section{System Calls}

A system call is a controlled entry from user mode to kernel mode.
User code triggers a trap using the \texttt{ecall} instruction, which saves the
current context and vectors to \texttt{syscall\_handler()}.

\begin{figure}[H]
\centering
\lstinputlisting[
  style=oscode,
  language=C,
  caption={System call handler, \texttt{syscall.c}},
  label={lst:syscall_c}
]{../code/chapter6/syscall.c}
\end{figure}

The current interface defines two calls:

\begin{center}
\begin{tabular}{ll}
  \texttt{SYS\_PUT} & Write a character to the display. \\
  \texttt{SYS\_GET} & Read a character from the keyboard (blocking). \\
\end{tabular}
\end{center}

\begin{figure}[H]
\centering
\lstinputlisting[
  style=oscode,
  language=C,
  caption={System call numbers and prototypes, \texttt{syscall.h}},
  label={lst:syscall_h}
]{../code/chapter6/syscall.h}
\end{figure}

The handler inspects the \texttt{a7} register to determine which system call was
invoked and uses the \texttt{trap\_frame} structure to access arguments and
return values.  
For now, calls simply wrap existing kernel routines like \texttt{proc\_put()} and
\texttt{uart\_get()}.

\section{User Library}

The user-side interface consists of small inline wrappers defined in
\texttt{syslib.h}.  
These issue the correct system-call number and execute \texttt{ecall}.

\begin{figure}[H]
\centering
\lstinputlisting[
  style=oscode,
  language=C,
  caption={Inline user-level system call stubs, \texttt{syslib.h}},
  label={lst:syslib_h}
]{../code/chapter6/syslib.h}
\end{figure}

\section{Demonstration Program}

The new demonstration (\texttt{hello.c}) creates several tasks that run in user
mode.  
Task~A repeatedly calls \texttt{user\_put()} to draw characters, while
Task~B uses \texttt{user\_get()} to echo typed input.
Each \texttt{ecall} traps into the kernel, which performs the corresponding
service and returns to user mode.

\begin{figure}[H]
\centering
\lstinputlisting[
  style=oscode,
  language=C,
  caption={Multitasking demonstration with system calls, \texttt{hello.c}},
  label={lst:hello_c}
]{../code/chapter6/hello.c}
\end{figure}

\section{Control Flow Summary}

\begin{enumerate}
  \item The kernel initializes interrupt vectors and system-call handling via
        \texttt{syscall\_init()} and \texttt{intr\_set\_handler(INTR\_SYSCALL,\dots)}.
  \item \texttt{ctx\_user\_setup()} programs PMP registers to permit user access.
  \item A thread enters user mode through \texttt{ctx\_start()}, which sets
        \texttt{mepc} to the entry function and executes \texttt{mret}.
  \item When user code executes \texttt{ecall}, the CPU traps into
        \texttt{syscall\_handler()}, which performs the requested service.
  \item On return, \texttt{mret} restores user mode and resumes at the next
        instruction.
\end{enumerate}

\section*{Projects}

\begin{enumerate}
  \item Extend the syscall interface with \texttt{SYS\_TIME} returning the current
        timer value, or \texttt{SYS\_YIELD} for explicit cooperative scheduling.
  \item Modify the user library to provide standard I/O functions like
        \texttt{putchar()} and \texttt{getchar()} using these system calls.
  \item Add simple error reporting: reserve negative return values for errors.
\end{enumerate}
