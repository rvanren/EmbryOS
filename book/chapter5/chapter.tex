\chapter{User Mode and a First System Call}

Up to now, all of EmbryOS has run in machine mode.  Processes, the scheduler,
and even \texttt{taskA()} all shared the same privilege.  This makes for simple
code, but it also means that a faulty process could scribble over the kernel or
hang the entire system.  Real operating systems prevent this by running
applications in \emph{user mode}, a restricted environment.  The kernel, still
in machine mode, exposes a controlled set of entry points called
\emph{system calls}.  This chapter makes that transition.

\section{Why User Mode?}

A privilege boundary gives us two essential properties:

\begin{itemize}
  \item \textbf{Isolation.} User code cannot modify kernel memory or device
        registers directly.  Faults in user programs become recoverable errors,
        not kernel panics.
  \item \textbf{Mediation.} All access to hardware---printing to the screen,
        reading input, allocating memory---goes through the kernel.  This lets
        the OS decide who can do what.
\end{itemize}

RISC-V defines three privilege levels: Machine~(M), Supervisor~(S), and
User~(U).  EmbryOS uses only two of them: M-mode for the kernel, and U-mode for
processes.  The CPU switches between them automatically whenever a \emph{trap}
occurs.

\section{How Traps Work}

When an interrupt or exception happens, the CPU saves the program counter in
\texttt{mepc}, records the cause in \texttt{mcause}, disables further
interrupts, and jumps to the address stored in \texttt{mtvec}.  Our assembly
handler \texttt{\_trap\_handler} (from earlier chapters) saves all registers on
the current stack and calls the C function
\texttt{software\_trap\_handler()}.  When that function returns, the assembly
code restores registers and executes \texttt{mret}.  The instruction
\texttt{mret} is what actually performs the privilege transition: it sets the
CPU's mode to whatever value is stored in the \texttt{mstatus.MPP} field and
resumes execution at \texttt{mepc}.

If the kernel clears \texttt{mstatus.MPP} to zero before executing
\texttt{mret}, control returns in user mode.  That is the key to launching user
programs.

\section{Dropping to User Mode}

The entry function \texttt{ctx\_user()} in Listing~\ref{lst:user_s} saves the
caller’s registers, stores the resulting stack pointer through its argument,
switches to the new process stack, sets \texttt{mepc} to the user entry
address, clears \texttt{mstatus.MPP} to select U-mode, and executes
\texttt{mret}.  The new process begins in user mode; the kernel remains dormant
until the next trap.

Because RISC-V does not provide memory protection by default, we use Physical
Memory Protection (PMP) to define which addresses user code may access.
\texttt{ctx\_user\_setup()} installs a NAPOT PMP region that starts at
0x80000000 and ends at 0x80400000, granting read, write, and execute
permissions within that 4\,MiB window.  Attempts to fetch or access memory
outside that range will cause an access fault.

\begin{figure}[H]
\centering
\lstinputlisting[
  style=oscode,
  language={[x86masm]Assembler},
  caption={Entering user mode and configuring PMP, \texttt{user.s}},
  label={lst:user_s}
]{../code/chapter5/user.s}
\end{figure}

\section{System Calls and ECALL}

Once in user mode, a process can no longer reach kernel symbols like
\texttt{proc\_put()}.  Instead it executes \texttt{ecall}, the
\emph{environment call} instruction.  This traps back into the kernel with
\texttt{mcause}=8 (ECALL from U-mode).  The handler runs in machine mode and
decides what to do.  The CPU then executes \texttt{mret} and the process
resumes where it left off.

\subsection*{A First Call: \texttt{SYS\_PUT}}

EmbryOS begins with just one system call: \texttt{SYS\_PUT}, which draws a
character in the process’s screen rectangle.  It takes five arguments:
row, column, character, foreground, and background color.

Listing~\ref{lst:syslib_h} shows the user stub \texttt{user\_put()}, which
places arguments in \texttt{a0--a4}, the system call number in \texttt{a7}, and
executes \texttt{ecall}.  Listing~\ref{lst:syscall_c} is the kernel’s handler.
It inspects \texttt{a7} to identify the system call, reads arguments from
\texttt{a0--a4}, finds the current process’s \texttt{pcb}, and calls
\texttt{proc\_put()} on its behalf.

\begin{figure}[H]
\centering
\lstinputlisting[
  style=oscode,
  language=C,
  caption={User-mode wrapper for \texttt{SYS\_PUT}, \texttt{syslib.h}},
  label={lst:syslib_h}
]{../code/chapter5/syslib.h}
\end{figure}

\begin{figure}[H]
\centering
\lstinputlisting[
  style=oscode,
  language=C,
  caption={System call implementation (one call), \texttt{syscall.c}},
  label={lst:syscall_c}
]{../code/chapter5/syscall.c}
\end{figure}

\begin{figure}[H]
\centering
\lstinputlisting[
  style=oscode,
  language=C,
  caption={System call numbers, \texttt{syscall.h}},
  label={lst:syscall_h}
]{../code/chapter5/syscall.h}
\end{figure}

\section{Traps Revisited: Timer and ECALL}

The trap dispatcher from Chapter~\ref{chap:preemption} now handles both
interrupts and exceptions.  A timer interrupt (\texttt{mcause}=0x80000007)
causes a preemption; the kernel resets the timer and calls the scheduler.
An ECALL from user mode (\texttt{mcause}=8) invokes the system call handler and
then increments \texttt{mepc} by~4 to skip the \texttt{ecall} instruction.
This prevents the process from re-trapping immediately upon return.

\begin{figure}[H]
\centering
\lstinputlisting[
  style=oscode,
  language=C,
  caption={Trap dispatcher with ECALL and timer handling, \texttt{interrupt.c}},
  label={lst:interrupt_c}
]{../code/chapter5/interrupt.c}
\end{figure}

\begin{figure}[H]
\centering
\lstinputlisting[
  style=oscode,
  language=C,
  caption={Interrupt interface, \texttt{interrupt.h}},
  label={lst:interrupt_h}
]{../code/chapter5/interrupt.h}
\end{figure}

\section{Putting It All Together}

The demonstration program (Listing~\ref{lst:hello_c}) initializes the timer and
interrupt subsystem, installs the ECALL handler, sets up PMP, and launches four
user-mode processes.  Each user process runs a loop that calls
\texttt{user\_put()} to draw moving letters in its window.  None of these
processes can access kernel data directly; all interaction goes through the
system call path:

\begin{enumerate}
  \item User executes \texttt{ecall}.
  \item CPU traps to machine mode, jumps to \texttt{\_trap\_handler}.
  \item \texttt{software\_trap\_handler()} calls \texttt{syscall\_handler()}.
  \item The kernel performs the service (here, \texttt{proc\_put()}).
  \item Kernel returns with \texttt{mret}; user resumes.
\end{enumerate}

This is the same basic sequence used by Unix, Linux, and every modern
operating system: a privilege change, a well-defined kernel entry, and a return
to user space.

\begin{figure}[H]
\centering
\lstinputlisting[
  style=oscode,
  language=C,
  caption={User-mode demo with preemption and \texttt{SYS\_PUT}, \texttt{hello.c}},
  label={lst:hello_c}
]{../code/chapter5/hello.c}
\end{figure}

\section*{Projects}
\begin{enumerate}
  \item \textbf{Add a second system call.}  Implement \texttt{SYS\_YIELD()} that
        calls the scheduler’s \texttt{yield()}.
  \item \textbf{Report errors.}  Make \texttt{SYS\_PUT()} return an error code in
        \texttt{a0} if the coordinates are outside the process’s rectangle.
  \item \textbf{Run-time privilege check.}  Print the value of
        \texttt{mstatus.MPP} at key points to verify the mode transitions.
\end{enumerate}
