\chapter{Hello World}

Before we dive into kernel code, let's set up a clean, reproducible environment.
This chapter will guide you through creating a Docker container that can build
and run our operating system, \textbf{EmbryOS}, on a RISC\textendash V
emulator.  By the end, we will make the machine print ``Hello~World'' using a
single UART register.

\section{Why Docker?}

Different computers have different toolchains and library versions.  Using a
container guarantees that everyone---on macOS, Windows, or Linux---is building
the same thing.  We will use Docker to install the RISC\textendash V compiler
and QEMU emulator inside a self-contained Ubuntu environment.

\section{Project Layout}

Each chapter comes with its own directory under \texttt{code/}:

\begin{verbatim}
code/
 +- chapter0/
 +- chapter1/
 |   +- Dockerfile
 |   +- Makefile
 |   +- main.c
 |   +- start.S
 |   \- linker.ld
 \- ...
\end{verbatim}

You can open these files in a simple text editor such as \texttt{nano} (already
included in the container) or your favorite local editor if you mount the
directory into Docker.

\section{The Dockerfile}

Before we write any kernel code, we need a consistent build and run environment.
To avoid platform differences (macOS vs. Linux, Intel vs. ARM), we will use
\texttt{Docker}.  Docker lets us define an entire Linux environment in a single
file—called a \texttt{Dockerfile}—that anyone can build and run.

Listing~\ref{lst:dockerfile} shows the complete Dockerfile for EmbryOS.  It
creates a container that installs all necessary tools: a RISC\textendash V
cross-compiler, the QEMU emulator, and a simple text editor (\texttt{nano}).
Once built, this container lets you compile and run EmbryOS on any machine.

% --- Dockerfile listing float ---
\begin{figure}[H]
\centering
\begin{minipage}{0.95\textwidth}
\begin{lstlisting}[style=oscode,language={},caption={Dockerfile for building and running EmbryOS on macOS/arm64},label={lst:dockerfile}]
# ------------------------------------------------------------
# Dockerfile for building and running EmbryOS on macOS/arm64
# ------------------------------------------------------------
FROM ubuntu:22.04

ENV DEBIAN_FRONTEND=noninteractive
ENV EMBRYOS=/workspace
WORKDIR $EMBRYOS

# Install build tools, RISC-V toolchain, QEMU, ...
RUN apt-get update && apt-get install -y --no-install-recommends \
    build-essential \
    gcc \
    make \
    git \
    wget \
    zip \
    unzip \
    gdb-multiarch \
    ca-certificates \
    nano \
    && apt-get clean && rm -rf /var/lib/apt/lists/*

# Clone EmbryOS
RUN git clone https://github.com/rvanren/EmbryOS.git

# Setup the RISC-V compiler
RUN wget https://github.com/xpack-dev-tools/riscv-none-elf-gcc-xpack/releases/download/v14.2.0-3/xpack-riscv-none-elf-gcc-14.2.0-3-linux-arm64.tar.gz && \
    tar -xzf xpack-riscv-none-elf-gcc-14.2.0-3-linux-arm64.tar.gz
ENV PATH="${PATH}:${EMBRYOS}/xpack-riscv-none-elf-gcc-14.2.0-3/bin"

# Setup QEMU
RUN wget https://github.com/xpack-dev-tools/qemu-riscv-xpack/releases/download/v8.2.2-1/xpack-qemu-riscv-8.2.2-1-linux-arm64.tar.gz && \
    tar -xzf xpack-qemu-riscv-8.2.2-1-linux-arm64.tar.gz
ENV PATH="${PATH}:${EMBRYOS}/xpack-qemu-riscv-8.2.2-1/bin"

CMD ["/bin/bash"]
\end{lstlisting}
\end{minipage}
\end{figure}

To build the container, run:

\begin{lstlisting}[style=oscode,language=bash]
$ docker build -t embryos-riscv .
\end{lstlisting}

The \texttt{-t} flag names the image \texttt{embryos-riscv}.  
Once the build finishes, you can start a container using:

\begin{lstlisting}[style=oscode,language=bash]
$ docker run -it --rm embryos-riscv
\end{lstlisting}

This command:
\begin{itemize}
  \item \texttt{-it} starts an interactive shell,
  \item \texttt{--rm} removes the container when you exit,
  \item \texttt{-v \$(pwd):/workspace} mounts your current directory inside the container.
\end{itemize}

When the shell prompt appears, you are inside a clean Ubuntu system with all the
tools installed.  You can edit files using \texttt{nano}, compile with
\texttt{make}, and run QEMU directly.

\bigskip
\noindent
This environment will be reused throughout the book; once you’ve built this
Docker image, you’re ready for every chapter that follows.

\section{The Makefile}

The next step is to tell the build system how to compile and run our first
program.  The \texttt{Makefile} below automates the process of compiling our
assembly and C code, linking it into a RISC\textendash V executable, and running
it in QEMU.

\begin{figure}[H]
\centering
\begin{minipage}{0.95\textwidth}
\begin{lstlisting}[style=oscode,language=make,caption={Makefile for building and running the ``Hello World'' kernel},label={lst:makefile}]
QEMU        = qemu-system-riscv32
RISCV_CC    = riscv-none-elf-gcc
OBJDUMP     = riscv-none-elf-objdump
OBJCOPY     = riscv-none-elf-objcopy
YELLOW      = \033[1;33m
END         = \033[0m

LDFLAGS     = -nostdlib -lc -lgcc
CFLAGS      = -march=rv32ima_zicsr -mabi=ilp32 -Wl,--gc-sections \
              -ffunction-sections -fdata-sections -fdiagnostics-show-option \
              -fno-builtin
DEBUG_FLAGS = --source --all-headers --demangle --line-numbers --wide

all:
	@printf "$(YELLOW)-------- Compile Hello, World! --------$(END)\n"
	$(RISCV_CC) $(CFLAGS) hello.s hello.c -Thello.lds $(LDFLAGS) -o hello.elf
	$(OBJDUMP) $(DEBUG_FLAGS) hello.elf > hello.lst
	$(OBJCOPY) -O binary hello.elf hello.bin

qemu: all
	@printf "$(YELLOW)-------- Run Hello-World on QEMU --------$(END)\n"
	$(QEMU) -nographic -machine sifive_u -smp 2 -bios hello.bin

clean:
	rm -f hello.bin hello.lst hello.elf
\end{lstlisting}
\end{minipage}
\end{figure}

The default \texttt{make} target (\texttt{all}) compiles both
\texttt{hello.s} (assembly) and \texttt{hello.c} (C code), then links them
using \texttt{hello.lds} (a linker script you’ll see shortly).  The result is an
executable file called \texttt{hello.elf}, along with a listing
(\texttt{hello.lst}) and a raw binary image (\texttt{hello.bin}) suitable for
booting on QEMU.

To run the system, type:

\begin{lstlisting}[style=oscode,language=bash]
$ make qemu
\end{lstlisting}

QEMU will start, emulate a RISC\textendash V processor, and execute
\texttt{hello.bin}.  You’ll see a simple console that prints your
``Hello World'' message.

\medskip
\noindent
\textbf{Exiting QEMU:}  
QEMU does not respond to \texttt{Ctrl+C}.  
To exit, press \texttt{Ctrl+A} followed by \texttt{x}.

\medskip
The \texttt{clean} target removes all generated files and lets you start fresh:

\begin{lstlisting}[style=oscode,language=bash]
$ make clean
\end{lstlisting}

In later chapters we will expand this Makefile to build a complete kernel and
user environment, but for now it provides a clear view of each step in the
toolchain—from C and assembly to executable and emulated hardware.

\section{The Code: Assembly and C}

Our ``Hello World'' kernel consists of two tiny source files: one in assembly
that sets up the machine and calls into C, and one in C that writes characters
to the UART device.  Together they form the simplest possible operating system.

\subsection{Assembly: Starting Up the Machine}

The RISC\textendash V processor begins execution at address zero in machine
mode.  The short assembly program in Listing~\ref{lst:hello_s} defines the entry
point \texttt{\_start}.  It disables the first core, initializes the stack
pointer, and then calls \texttt{main()} written in C.

\begin{figure}[H]
\centering
\begin{minipage}{0.9\textwidth}
\begin{lstlisting}[style=oscode,language={[x86masm]Assembler},caption={Startup assembly for EmbryOS},label={lst:hello_s}]
    .section .text.enter
    .global _start

_start:
    csrr a0, mhartid
    beq a0, x0, _end        /* disable core zero */
    li sp, 0x80400000       /* set stack pointer */
    call main
_end:
    j _end                  /* loop forever */
\end{lstlisting}
\end{minipage}
\end{figure}

A few details:

\begin{itemize}
  \item \texttt{csrr a0, mhartid} reads the hardware thread ID (``hart ID'') of
        the running core.
  \item Core~0 is disabled so that only the second core executes the program;
        this simplifies early testing.
  \item The instruction \texttt{li sp, 0x80400000} initializes the stack
        pointer to an arbitrary address in memory.
  \item Finally, \texttt{call main} jumps into the C function
        \texttt{main()}, after which control will not return.
\end{itemize}

This short sequence is all that’s needed to transition from the CPU’s reset
state into C code.

\subsection{C: Talking to the UART}

Once the processor is running C code, we can access hardware devices directly
using \emph{memory-mapped I/O}.  The UART (serial port) is available at address
\texttt{0x10010000} on QEMU’s \texttt{sifive\_u} machine.  Writing a character
to this address sends it to the terminal.

\begin{figure}[H]
\centering
\begin{minipage}{0.95\textwidth}
\lstinputlisting[style=oscode,language=C,caption={C code for printing ``Hello World'' through the UART},label={lst:ch1_hello_c}]{../code/chapter1/hello.c}
\end{minipage}
\end{figure}

Here, the compiler treats the \texttt{UART} pointer as a direct reference to a
hardware device.  The \texttt{putchar()} function busy-waits until the UART is
ready, then writes a single character.  \texttt{printf()} loops through a string
and sends each character one at a time.

\medskip
\noindent
When QEMU runs this program, each call to \texttt{putchar()} writes to the UART
register, which appears on your terminal.  The result is the familiar output:

\begin{lstlisting}
Hello World
\end{lstlisting}

\medskip
\noindent
This is a complete operating system in fewer than thirty lines of code: it
boots, runs C, communicates with hardware, and never returns.

\section{The Linker Script}

The last piece of our ``Hello World'' system is the linker script,
\texttt{hello.lds}.  A linker script tells the linker how to arrange the
different sections of code and data in memory.  Without it, the compiler would
produce object files but not know where to place them in physical memory.

Listing~\ref{lst:hello_lds} shows the script we use for EmbryOS.  It declares a
single region of memory, starting at \texttt{0x80000000}, and places all code
and data there.  The \texttt{\_start} symbol defined in the assembly file is set
as the program’s entry point.

\begin{figure}[H]
\centering
\begin{minipage}{0.95\textwidth}
\begin{lstlisting}[style=oscode,language={},caption={Linker script \texttt{hello.lds} for EmbryOS},label={lst:hello_lds}]
OUTPUT_ARCH("riscv")

ENTRY(_start)

MEMORY
{
    ram (arw!xi) : ORIGIN = 0x80000000, LENGTH = 0x200000
}

PHDRS
{
    ram PT_LOAD;
}

SECTIONS
{
    .init : ALIGN(8) {
        *(.text.enter)
    } >ram :ram

    .text : ALIGN(8) {
        *(.text .text.*)
    } >ram :ram

    .rodata : ALIGN(8) {
        *(.rdata)
        *(.rodata .rodata.*)
        . = ALIGN(8);
        *(.srodata .srodata.*)
    } >ram :ram

    .data : ALIGN(8) {
        *(.data .data.*)
        . = ALIGN(8);
        *(.sdata .sdata.* .sdata2.*)
    } >ram :ram

    .bss (NOLOAD): ALIGN(8) {
        *(.sbss*)
        *(.bss .bss.*)
        *(COMMON)
    } >ram :ram

    .heap (NOLOAD) : ALIGN(8) {
        PROVIDE( __heap_start = . );
    } >ram :ram

    PROVIDE( __heap_end = 0x80200000 );
}
\end{lstlisting}
\end{minipage}
\end{figure}

Here is what the main parts mean:

\begin{itemize}
  \item \textbf{\texttt{MEMORY}} defines a single block of RAM starting at
        \texttt{0x80000000}.  This matches the QEMU \texttt{sifive\_u} machine’s
        default memory map.
  \item \textbf{\texttt{ENTRY(\_start)}} tells the linker that the program should
        begin at the symbol \texttt{\_start}, which we defined in the assembly
        file.
  \item Each \textbf{section} (\texttt{.text}, \texttt{.rodata},
        \texttt{.data}, \texttt{.bss}) gathers code and data from all compiled
        object files and places them sequentially in RAM.
  \item \textbf{Alignment directives} (\texttt{ALIGN(8)}) ensure proper
        alignment for 64-bit accesses, even though we are running a 32-bit CPU.
  \item The \textbf{heap symbols} \texttt{\_\_heap\_start} and
        \texttt{\_\_heap\_end} reserve a small area for dynamic allocation later.
\end{itemize}

At this point, the entire system layout is defined:
\begin{center}
\begin{tabular}{ll}
\texttt{0x80000000} & --- code (\texttt{.text}, \texttt{.rodata}) \\
\texttt{0x80200000} & --- heap end (\texttt{\_\_heap\_end}) \\
\texttt{0x80400000} & --- stack top (set in assembly)
\end{tabular}
\end{center}

\medskip
\noindent
When the linker combines all object files according to this script, the result
is a single binary image (\texttt{hello.bin}) that QEMU can load directly into
RAM and start executing from \texttt{0x80000000}.

\bigskip
\noindent
This completes Chapter~1.  You now have:
\begin{itemize}
  \item A Docker environment that builds EmbryOS reproducibly;
  \item A Makefile that compiles, links, and runs the system;
  \item Minimal assembly and C code that boot the machine and print text;
  \item A linker script that defines how the program fits into memory.
\end{itemize}

