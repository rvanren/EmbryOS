\chapter{What This Book Is About}

This book is about building an operating system entirely from scratch for the
RISC\textendash V architecture.  Not by linking in large existing frameworks,
not by copying Linux, and not by running in a simulator with hidden support
code—but by writing code that truly starts at address zero and brings a
bare-metal RISC\textendash V processor to life.

Our goal is educational, not industrial.  Each chapter introduces one small
concept, shows how to implement it, and builds on previous chapters.  By the end
you will understand how the key components of an operating system—device
drivers, timers, interrupts, memory management, processes, and file systems—fit
together and interact with hardware.

The philosophy of this book is incremental construction.  In each chapter we
will:
\begin{enumerate}
  \item Introduce a new concept.
  \item Describe the relevant hardware interface.
  \item Show minimal working code that exercises it.
  \item Explain the reasoning and design trade-offs.
\end{enumerate}

We start with the simplest possible goal: make the machine say
``Hello~World''.  To do this we will write directly to a UART register.
Then, step by step, we will add timer interrupts, handle exceptions, switch
between tasks, and eventually build an environment capable of running user
programs.

To keep things transparent, we avoid black boxes.  The operating system will be
written in C (with occasional assembly), and the code will run on both
emulators such as QEMU and real RISC\textendash V boards.

Each chapter is designed to be self-contained: by the end of it, you should
have a working system that boots, runs the example, and can be tested and
modified.  The chapters are cumulative but modular—you can start at any point
if you already know the prerequisites.

\bigskip
\noindent
\textbf{Who this book is for:} Students, hobbyists, and systems programmers who
want to learn what happens below the compiler, the kernel, and the libraries.

\bigskip
\noindent
\textbf{What you need:}
\begin{itemize}
  \item A basic understanding of C and assembly.
  \item A RISC\textendash V toolchain and QEMU (we will show how to install them).
  \item Curiosity and patience.
\end{itemize}

\bigskip
By the time we are done, you will not only have built a small but complete
operating system—you will have demystified how computers actually start,
communicate, and manage themselves.
